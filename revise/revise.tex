%        File: revise.tex
%     Created: Wed Oct 09 02:00 PM 2013 P
% Last Change: Wed Oct 09 02:00 PM 2013 P
%

%
% Copyright 2007, 2008, 2009 Elsevier Ltd
%
% This file is part of the 'Elsarticle Bundle'.
% ---------------------------------------------
%
% It may be distributed under the conditions of the LaTeX Project Public
% License, either version 1.2 of this license or (at your option) any
% later version.  The latest version of this license is in
%    http://www.latex-project.org/lppl.txt
% and version 1.2 or later is part of all distributions of LaTeX
% version 1999/12/01 or later.
%
% The list of all files belonging to the 'Elsarticle Bundle' is
% given in the file `manifest.txt'.
%

% Template article for Elsevier's document class `elsarticle'
% with numbered style bibliographic references
% SP 2008/03/01
%
%
%
% $Id: elsarticle-template-num.tex 4 2009-10-24 08:22:58Z rishi $
%
%
%\documentclass[preprint,12pt]{elsarticle}
\documentclass[answers,11pt]{exam}

% \documentclass[preprint,review,12pt]{elsarticle}

% Use the options 1p,twocolumn; 3p; 3p,twocolumn; 5p; or 5p,twocolumn
% for a journal layout:
% \documentclass[final,1p,times]{elsarticle}
% \documentclass[final,1p,times,twocolumn]{elsarticle}
% \documentclass[final,3p,times]{elsarticle}
% \documentclass[final,3p,times,twocolumn]{elsarticle}
% \documentclass[final,5p,times]{elsarticle}
% \documentclass[final,5p,times,twocolumn]{elsarticle}

% if you use PostScript figures in your article
% use the graphics package for simple commands
% \usepackage{graphics}
% or use the graphicx package for more complicated commands
\usepackage{graphicx}
% or use the epsfig package if you prefer to use the old commands
% \usepackage{epsfig}

% The amssymb package provides various useful mathematical symbols
\usepackage{amssymb}
% The amsthm package provides extended theorem environments
% \usepackage{amsthm}
\usepackage{amsmath}

% The lineno packages adds line numbers. Start line numbering with
% \begin{linenumbers}, end it with \end{linenumbers}. Or switch it on
% for the whole article with \linenumbers after \end{frontmatter}.
\usepackage{lineno}

% I like to be in control
\usepackage{placeins}

% natbib.sty is loaded by default. However, natbib options can be
% provided with \biboptions{...} command. Following options are
% valid:

%   round  -  round parentheses are used (default)
%   square -  square brackets are used   [option]
%   curly  -  curly braces are used      {option}
%   angle  -  angle brackets are used    <option>
%   semicolon  -  multiple citations separated by semi-colon
%   colon  - same as semicolon, an earlier confusion
%   comma  -  separated by comma
%   numbers-  selects numerical citations
%   super  -  numerical citations as superscripts
%   sort   -  sorts multiple citations according to order in ref. list
%   sort&compress   -  like sort, but also compresses numerical citations
%   compress - compresses without sorting
%
% \biboptions{comma,round}

% \biboptions{}


% Katy Huff addtions
\usepackage{xspace}
\usepackage{color}
\usepackage[hyphens]{url}

%\journal{Annals of Nuclear Energy}

\begin{document}

%\begin{frontmatter}

   % Title, authors and addresses

   % use the tnoteref command within \title for footnotes;
   % use the tnotetext command for the associated footnote;
   % use the fnref command within \author or \address for footnotes;
   % use the fntext command for the associated footnote;
   % use the corref command within \author for corresponding author footnotes;
   % use the cortext command for the associated footnote;
   % use the ead command for the email address,
   % and the form \ead[url] for the home page:
   %
   % \title{Title\tnoteref{label1}}
   % \tnotetext[label1]{}
   % \author{Name\corref{cor1}\fnref{label2}}
   % \ead{email address}
   % \ead[url]{home page}
   % \fntext[label2]{}
   % \cortext[cor1]{}
   % \address{Address\fnref{label3}}
   % \fntext[label3]{}

\title{Introduction to Moltres: an Application for Simulation of Molten Salt 
        Reactors}

   % use optional labels to link authors explicitly to addresses:
   % \author[label1,label2]{<author name>}
   % \address[label1]{<address>}
   % \address[label2]{<address>}


%\author[uiuc]{Kathryn Huff}
%        \ead{kdhuff@illinois.edu}
%  \address[uiuc]{Department of Nuclear, Plasma, and Radiological Engineering,
%        118 Talbot Laboratory, MC 234, Universicy of Illinois at
%        Urbana-Champaign, Urbana, IL 61801}
%
% \end{frontmatter}

\section*{Review General Response}
We would like to again thank the reviewers for their detailed assessment of 
this paper. 


\section*{Reviewer 1}
\begin{questions}

\question The paper presents an interesting and original work on a new solver
for MSRs based on the MOOSE platform.  It is clearly written but it lacks depth
both in the description of the work and in the positioning of it in the
available scientific literature.  In addition, the paper fails in properly
supporting the main claim of an ``unmatched fidelity in coupled neutronics and
thermal hydraulics MSR simulation'', since the proposed results are limited and
compared only with an old MSRE model. The paper would represent an interesting
contribution to the scientific community but it requires revision. In
particular the authors should deepen their bibliographical research, and
include a comparison either with experimental results or with other
state-of-the-art tools.
\begin{solution}
        Thank you for your kind review.
        The statement concerning `unmatched' fidelity has been removed 
        entirely and the bibliography has been expanded significantly. See the 
        responses below for more detail..
\end{solution}


\question In the introduction, the authors mention that:
``Two key advantages offered by the fluid-fuelled MSR are improved fuel 
        utilization and no radiation damage constraint on attainable fuel 
        burn-up. Together, these attributes result in significantly reduced 
        core fissile inventories and spent nuclear fuel mass''
However, the fissile inventory is not necessarily related to fuel burnup and fuel utilization. It can actually be  very high in chloride-based fast-spectrum MSRs.
\begin{solution}
        <++>
\end{solution}

\question The introduction includes a review of previous work done on the subject. However, few important (and recent) works have not been mentioned. In particular, significant efforts have been spent in the frame of the Euratom Evol project, and the results have been published in several papers, including e.g.,:

Manuele Aufiero, Antonio Cammi, Olivier Geoffroy, Mario Losa, Lelio Luzzi, Marco E. Ricotti, Hervé Rouch, Development of an OpenFOAM model for the Molten Salt Fast Reactor transient analysis, Chemical Engineering Science, Volume 111, 2014, Pages 390-401, ISSN 0009-2509, http://dx.doi.org/10.1016/j.ces.2014.03.003.
(http://www.sciencedirect.com/science/article/pii/S0009250914001146)

Carlo Fiorina, Danny Lathouwers, Manuele Aufiero, Antonio Cammi, Claudia Guerrieri, Jan Leen Kloosterman, Lelio Luzzi, Marco Enrico Ricotti, Modelling and analysis of the MSFR transient behaviour, Annals of Nuclear Energy, Volume 64, 2014, Pages 485-498, ISSN 0306-4549, http://dx.doi.org/10.1016/j.anucene.2013.08.003.
(http://www.sciencedirect.com/science/article/pii/S0306454913004118)

Matteo Zanetti, Antonio Cammi, Carlo Fiorina, Lelio Luzzi, A Geometric Multiscale modelling approach to the analysis of MSR plant dynamics, Progress in Nuclear Energy, Volume 83, 2015, Pages 82-98, ISSN 0149-1970, http://dx.doi.org/10.1016/j.pnucene.2015.02.014.

A. Laureau, D. Heuer, E. Merle-Lucotte, P.R. Rubiolo, M. Allibert, M. Aufiero, Transient coupled calculations of the Molten Salt Fast Reactor using the Transient Fission Matrix approach, Nuclear Engineering and Design, Volume 316, 2017, Pages 112-124, ISSN 0029-5493, http://dx.doi.org/10.1016/j.nucengdes.2017.02.022.

Other important works have been prepared in China in the frame of the Chinese MSR program.

The authors should deepen the discussion on the current state of the art, mentioning at least the several efforts spent in the Evol project, and re-balancing their claims accordingly.
\begin{solution}
        Thank you for the excellent recommendations. The new text reads thus:


        Simulation tools developed by many authors successfully describe 
        steady-state and
        transient behavior of myriad MSR concepts. Krepel et al. extended 
        the in-house LWR
        diffusion code DYN3D to consider drift of delayed neutron precursors 
        alongside
        the reactor temperature profile, re-casting the extended code as
        DYN3D-MSR \cite{krepel_dyn3d-msr_2007}. That work compared DYN3D-MSR 
        against
        experimental MSRE data and then used it to simulate local fuel 
        channel
        blockages as well as local temperature perturbations. 

        In a similar vein, Kophazi et al. used iterative coupling between 
        in-house
        three-dimensional neutronic and one-dimensional heat conduction models 
        DALTON
        and THERM to analyze normal MSRE operation as well as
        channel-blocking-incident transients \cite{kophazi_development_2009}. 
        The
        Kophazi model added entrance effects of heat transfer coefficients as 
        well as
        thermal coupling between fuel channels through moderator heat 
        conduction. More
        recently, Cammi et al. performed a 2D-axisymmetric single-channel 
        analysis of
        the MSBR using the commercial finite element package COMSOL 
        Multiphysics
        \cite{cammi_multi-physics_2011}. That work directly solved the fuel 
        salt
        velocity field, used heterogeneous group constants in fuel and 
        moderator
        regions, and employed the COMSOL software package intrinsically 
        designed
        for coupled multi-physics simulation.  Fiorina, Lathouwers, and their
        colleagues conducted a benchmarking exercise 
        \cite{fiorina_modelling_2013} in
        which this Politecnico di Milano approach was expanded to a 
        multi-channel model 
        of the MSFR and compared to code from the University of Delft
        \cite{de_zwaan_static_2007,van_der_linden_2012} based on the approach 
        in
        \cite{kophazi_development_2009}. These models showed good agreement for
        multiple accident transients. Meanwhile, leveraging lessons learned 
        from these 
        efforts has resulted in a multiscale approach from Zanetti et al. 
        \cite{zanetti_geometric_2015} successfully combines high and low 
        geometric 
        fidelity for graphite-moderated MSRs.

        More recently, Aufiero et al. \cite{aufiero_development_2014} have 
        begun to
        approach transient simulations in the MSFR within the finite 
        volume
        OpenFOAM multiphysics toolkit \cite{weller_tensorial_1998}.  This 
        approach
        benefits from pre-implemented turbulence models available in the 
        OpenFOAM
        library and captures the full-core three-dimensional geometry of the 
        reactor
        primary circuit.  OpenFOAM CFD has additionally been shown by 
        Laureau et
        al. \cite{laureau_transient_2017} to couple well with Transient Fission 
        Matrix
        neutronics within the MSFR.
\end{solution}

\question The authors write:
``Moltres is devoted to previously unmatched fidelity in coupled neutronics and 
        thermal hydraulics MSR simulation.''
This is an overstatement as codes with similar degree of fidelity have been written in the past. This is the case of the tools developed in the frame of the Evol project, as well as e.g. the work from Aufiero on the coupling between OpenFOAM and Serpent (that the authors mention earlier in the introduction). The authors are required to modify this sentence based on a better bibliographical research.
\begin{solution}
        That statement has been removed and the bibliography has been enriched 
        (as discussed above).
\end{solution}

\question Concerning the openness of the code and the possibility to contribute to it, it should be noted that codes solving for neutron transport can be subject to export restrictions (following their potential dual use), and in principle all contributors should get an export control authorization from their own country. The authors are not strictly required to change the manuscript, but they may consider mentioning the fact that the openness of a nuclear code can be somewhat thwarted by national and international laws.
\begin{solution}
	Excellent point. We have added the following text: ``While export
control laws have the potential to restrict openness of neutron transport
software that is dual use, there is no reason to believe that Moltres could be
used for weapons proliferation.  Moltres requires that cross sections be
provided by the user, implements a combination of algorithms that can be easily
derived from open literature, is tailored to molten salt reactor physics.
Accordingly, Moltres joins a veritable parade of open academic nuclear
engineering software such as OpenMC \cite{romano_openmc:_2015},
OpenMOC \cite{boyd_openmoc_2014}, and
PyNE \cite{bates_pyne_2014,biondo_quality_2014}.'' 

\end{solution}

\question It's somehow interesting that ``For some three-dimensional 
simulations, the number of elements in the mesh and total number of degrees of 
freedom exceed one million and ten million respectively. To handle problems of 
this size, we ran Moltres on up to 608 cores.'' Normally problems of few million degrees of freedom can be handled with few tens of cores. The authors should discuss more extensively the performances of their code in order to give the reader the possibility to compare them with those of other tools. In particular, it is possible that slow performances may derive from trying to achieve a fully coupled solution (i.e., all the equations in the same matrix), which seems to be the case in this work. In this sense and to better support the claims of the paper, it would be important to compare the performances of a segregated and a fully coupled  solver, both in terms of accuracy and computational time. In addition to helping supporting the claims of the paper, this
would represent a significant contribution to the scientific community, since many of the currently available tools employ segregated solvers, and since the advantages and disadvantages of a fully coupled solver are often a subject of discussion.
\begin{solution}
        <++>
\end{solution}


\question Are all equations solved on the same mesh? In this case, is this a strict requirement? Please clarify.
\begin{solution}
        This has now been clarified in the introductory text. The equations are 
        solved on a single mesh. This is enabled by the underlying MOOSE 
        framework, which also enables `loose' coupling instead.
\end{solution}

\question The heat generation in the graphite is calculated as a fraction of that in the fuel. How exactly is this achieved considering that the two sources are spatially separated?
\begin{solution}
        <++>
\end{solution}

\question It is not clear how the domain is discretized. Is there a radial discretization in graphite and channels? Was the mesh convergence checked? Please discuss.
\begin{solution}
       <++>
\end{solution}

\question The MSRE had Reynolds numbers very close to the laminar-turbulent transition. Did you use a laminar model? In case, did you encounter convergence problems ?
Or did you include turbulence? More generally, is Moltres capable of turbulent simulations?
\begin{solution}
        These calculations were performed without including turbulence. However, a $k-\epsilon$ model is being implemented to support future simulations.
\end{solution}

\question The authors are required to use the international system of units (meters, Kelvin, etc).
\begin{solution}
        Thanks for catching this. The mention of inches has been converted to cm.
\end{solution}

\question The authors should briefly describe what ``MSRE design models'' are. Otherwise it is very difficult for the readers to assess the accuracy of the model.
\begin{solution}
        The paper has been clarified. That section is now careful to state that 
        our comparison is made against a 1963-1964 calculation conducted by ORNL 
        to pinpoint the MSRE design. The text has been updated to refer to 
        ``MSRE design models'' as ``MSRE design calculations'' and explains 
        that the MSRE design calculations conducted in 1963 were 32-group 
        calculations using legacy computing tools (GAM-I, MODRIC, and EQUIPOSE, 
        and THERMOS). These calculations were conducted in two-dimensional R-Z 
        geometry (a cylinder with angular symmetry), with 20 spatial regions. 
        Notably, one limitation of the ORNL model was the control rod thimbles 
        that, due to angular symmetry, were effectively a cylindrical shell of 
        metal.  
\end{solution}

\question Is the heat generation in the hottest channel the same for the Moltres and the MSRE model?
\begin{solution}
        <++>
\end{solution}

\question The authors claim an unprecedented accuracy of their model? How the results of their model compare with previous results from other authors? The presented results do not seem sufficient to support such claim.
\begin{solution}
        This claim has been removed entirely. Thank you.
\end{solution}

\question The differences between the MSRE and the Moltres model are significant and should be discussed in more details. In particular the difference between fast fluxes is suspicious.
\begin{solution}
        <++>
\end{solution}

\question The proposed scaling test is very limited. Proper scaling tests would require weak and strong scaling tests, an assessment on the impact of the size of problem per core, a separate assessment of intra-node and extra-node scaling, etc. In addition, information should be provided on the employed numerical algorithms and on the parallelization strategy. The authors should either extend significantly the section, or remove it completely.
\begin{solution}
        We performed strong and weak scaling up to 768 cores separately for 
        intra-node and extra-node cases. Parallelization was implemented using 
        LibMesh library for automatic domain decomposition and PetSc 
        library for parallel solution of systems of non-linear equations. 
        Detailed information, including results of the scaling studies, has now 
        been added in the Section 4.3 introduction.
\end{solution}


\end{questions}
\bibliographystyle{unsrt}
\bibliography{Moltres}
  \end{document}

  %
  % End of file `elsarticle-template-num.tex'.


